%\def\pdfliteral#1{\special{pdf:content #1}}%pict2e用?
\usepackage[dvipdfmx]{graphicx}
\usepackage[usenames]{color}
\usepackage[deluxe]{otf}
\setlength{\unitlength}{1mm}
%\setlength{\voffset}{0mm}
%\setlength{\hoffset}{0mm}
\setlength{\topmargin}{-1in}
\addtolength{\topmargin}{19mm}
\setlength{\oddsidemargin}{-1in}
\addtolength{\oddsidemargin}{10.5mm}
\setlength{\evensidemargin}{-1in}
\addtolength{\evensidemargin}{10.5mm}
\setlength{\textwidth}{276mm}
\setlength{\textheight}{171mm}
\setlength{\headsep}{0mm}
\setlength{\headheight}{0mm}
\setlength{\footskip}{19mm}
\setlength{\fboxsep}{0mm}
\setlength{\parindent}{0pt}
\setlength{\baselineskip}{0pt}
\linethickness{1pt}
%
\makeatletter
%%%%%%%%%%%%%%%%%%%%%%%%%%%%%%%%%%%%%%%%%%%%%%%%%%%%%%%%%%%%%%%%%%%%%%
%均等割
%%%%%%%%%%%%%%%%%%%%%%%%%%%%%%%%%%%%%%%%%%%%%%%%%%%%%%%%%%%%%%%%%%%%%%
\newcommand{\kintou}[2]{%
	\leavevmode
	\hbox to #1{%
		\kanjiskip=0pt plus 1fill minus 1fill
		\xkanjiskip=\kanjiskip
		#2}}
%%%%%%%%%%%%%%%%%%%%%%%%%%%%%%%%%%%%%%%%%%%%%%%%%%%%%%%%%%%%%%%%%%%%%%
%文字数カウント
%%%%%%%%%%%%%%%%%%%%%%%%%%%%%%%%%%%%%%%%%%%%%%%%%%%%%%%%%%%%%%%%%%%%%%
\newcommand{\wordCount}[1]{%
  \@tempcnta\z@
  \@tfor \@tempa:=#1\do{\advance\@tempcnta\@ne}%
  \the\@tempcnta
}
%%%%%%%%%%%%%%%%%%%%%%%%%%%%%%%%%%%%%%%%%%%%%%%%%%%%%%%%%%%%%%%%%%%%%%
%個人枠
%%%%%%%%%%%%%%%%%%%%%%%%%%%%%%%%%%%%%%%%%%%%%%%%%%%%%%%%%%%%%%%%%%%%%%
\newcommand{\person}[3]{%
	%敬称の基準長さ
	\setbox4=\hbox{{\fontsize{\parsonKeisyouFSize}{\parsonKeisyouFSize}\selectfont 様}}
	\setbox5=\hbox{{\fontsize{\parsonKeisyouFSize}{\parsonKeisyouFSize}\selectfont #3}}
	\parsonKeisyouFSize=\parsonKeisyouResetFSize
	\parsonWidth=\parsonResetWidth
	\ifdim\wd5>\wd4%「くん」や「ちゃん」の処理
	\advance\parsonKeisyouFSize-2pt%文字サイズを小さく
	\setbox6=\hbox{{\fontsize{\parsonKeisyouFSize}{\parsonKeisyouFSize}\selectfont #3}}
	\toks4{\resizebox{5.5mm}{\ht6}{#3}}%様の幅に合わせる
	\advance\parsonWidth by 2mm
	\else
	\toks4{#3}
	\fi
	
	%名前の基準長さ
	\setbox0=\hbox{{\fontsize{\parsonNameFSize}{\parsonNameFSize}\selectfont ○○○○○○ \the\toks4}}
	%実際の名前の長さ
	\setbox1=\hbox{{\fontsize{\parsonNameFSize}{\parsonNameFSize}\selectfont #2 \the\toks4}}
	\dimen0=1pt
	\ifdim\wd1>\wd0%
		\toks0{\resizebox{\wd0}{\ht0}{#2{\thinspace}\the\toks4}}
	\else
		\ifdim\wd1<\wd0%
			\toks0{\kintou{6zw}{#2} \the\toks4}
		\else
			\toks0{#2 \the\toks4}
		\fi
	\fi
	%肩書きの基準高さ
	\parsonKatagakiFSize=\parsonKatagakiResetFSize%初期化
	\setbox2=\hbox{\parbox[c]{\parsonWidth}{\fontsize{\parsonKatagakiFSize}{\parsonKatagakiFSize}\selectfont #1}}
	
	
	\@whiledim\ht2>\katagakiHeight\do{%肩書きが2行を越える場合、文字サイズを小さくして2行に収める
	\advance\parsonKatagakiFSize-0.05pt
	\setbox2=\hbox{\parbox[c]{\parsonWidth}{\fontsize{\parsonKatagakiFSize}{\parsonKatagakiFSize}\selectfont #1}}
	%%\@tfor\setbox3:=#1\do{\scalebox{.5}[1]{}}
	}
	
	\toks1{\fontsize{\parsonKatagakiFSize}{\parsonKatagakiFSize}\selectfont #1}
	
	%\fbox{%
	\begin{minipage}[c][\parsonHeight][c]{\parsonWidth}%
	\setlength{\baselineskip}{0pt}
	\the\toks1\\[\nameKatagakiVspace]
	{\fontsize{\parsonNameFSize}{\parsonNameFSize}\selectfont \the\toks0}\\[0mm]
	\end{minipage}%}%
	\vspace\personVspace}%
%%%%%%%%%%%%%%%%%%%%%%%%%%%%%%%%%%%%%%%%%%%%%%%%%%%%%%%%%%%%%%%%%%%%%%
%個人枠1列分の集合
%%%%%%%%%%%%%%%%%%%%%%%%%%%%%%%%%%%%%%%%%%%%%%%%%%%%%%%%%%%%%%%%%%%%%%
\newcommand{\personsUnit}[1]{%
	%\fbox{%
	\begin{minipage}[c][\personsUnitHeight][c]{\personsUnitWidth}%
	#1%
	\end{minipage}}%}%
%%%%%%%%%%%%%%%%%%%%%%%%%%%%%%%%%%%%%%%%%%%%%%%%%%%%%%%%%%%%%%%%%%%%%%
%座席内側のグルーピング枠
%%%%%%%%%%%%%%%%%%%%%%%%%%%%%%%%%%%%%%%%%%%%%%%%%%%%%%%%%%%%%%%%%%%%%%
\newcommand{\seat}[1]{%
	\begin{minipage}[b][\seatHeight][t]{\seatWidth}%
	#1%
	\end{minipage}}%
%%%%%%%%%%%%%%%%%%%%%%%%%%%%%%%%%%%%%%%%%%%%%%%%%%%%%%%%%%%%%%%%%%%%%%
%座席ユニット
%%%%%%%%%%%%%%%%%%%%%%%%%%%%%%%%%%%%%%%%%%%%%%%%%%%%%%%%%%%%%%%%%%%%%%
\newcommand{\seatUnit}[3]{%
	\put(#1){%
	%\hskip-1.2mm%
	\tableUnit{39}{22}{#2}{0}%
	\fbox{%
	\begin{minipage}[b][\seatUnitHeight][c]{\seatUnitWidth}%
	%%%%%%%%%%%%%%%%%%%%%%%%
	%\hskip7.5mm%
	%%%%%%%%%%%%%%%%%%%%%%%%
	\centering{#3}%
	\end{minipage}}%
	}}%
%%%%%%%%%%%%%%%%%%%%%%%%%%%%%%%%%%%%%%%%%%%%%%%%%%%%%%%%%%%%%%%%%%%%%%
%テーブルの描画ルーチン
%%%%%%%%%%%%%%%%%%%%%%%%%%%%%%%%%%%%%%%%%%%%%%%%%%%%%%%%%%%%%%%%%%%%%%
\newcommand{\tableUnit}[4]{%
	\newdimen\xLength
	\newdimen\yLength
	\newdimen\tableX
	\newdimen\tableY
	\xLength=#1 pt
	\yLength=#2 pt
	\newdimen\tableY
	\tableX=#1 pt
	\tableY=#2 pt
	\ifnum#4=1
		%%長机
		\advance\xLength by -6.4pt
		\advance\yLength by -20pt%テーブルの長さの半分を引く
		\put(\strip@pt\xLength,\strip@pt\yLength){\framebox(11,40){%frameboxの第2引数がテーブルの長さ
		\fontsize{18pt}{18pt}\selectfont {\textsf{\textgt #3}}}}
	\else
		%%円卓
		\linethickness{0.3pt}%
		%
		%円卓中央に文字が来るよう微調整
		\advance\xLength by -2.7pt%
		\advance\yLength by -1.95pt%
		
		%実線の円卓(文字黒)
		%\put(#1,#2){\circle{9}}
		%\put(\strip@pt\xLength,\strip@pt\yLength){\fontsize{16pt}{16pt}\selectfont {\textsf{\textgt #3}}}
		
		%グラフィックの円卓(文字白)
		\advance\tableX by -2.3mm%
		\advance\tableY by -2.3mm%
		\put(\strip@pt\tableX,\strip@pt\tableY){\includegraphics{\imagePath/table/\tableColor.pdf}}
		\put(\strip@pt\xLength,\strip@pt\yLength){\fontsize{16pt}{16pt}\selectfont {\textsf{\textgt{\textcolor{white}{#3}}}}}
	\fi
	%%%\put(45.5,27){\fontsize{22pt}{22pt}\selectfont {\textsf{\textgt #3}}}
	}
%%%%%%%%%%%%%%%%%%%%%%%%%%%%%%%%%%%%%%%%%%%%%%%%%%%%%%%%%%%%%%%%%%%%%%
%高砂を含むヘッダブロック
%
%
%%%%%%%%%%%%%%%%%%%%%%%%%%%%%%%%%%%%%%%%%%%%%%%%%%%%%%%%%%%%%%%%%%%%%%
\newcommand{\headUnit}[6]{%
	\put(#1){%
	%ヘッダレイアウト枠
	%\fbox{%
	\begin{minipage}[b][30mm][c]{276mm}%
		\vskip-8mm%ヘッダ枠内の上アキの調整
		%席次表タイトルユニット
		\hskip25mm%
		\begin{minipage}[c][10mm][c]{60mm}%
		{\fontsize{11pt}{16pt}\selectfont %
		\parbox[c][10mm][c]{4.5zw}{%
		\kintou{4zw}{#2家}\\%
		\kintou{4zw}{#3家}%
		}結婚披露宴御席表}%
		\end{minipage}%
		%高砂ユニット
		\hskip25.5mm%
		%\fbox{%
		\begin{minipage}[c]{55mm}%
		\vskip1mm%
		\centering{%
		\parbox[c][11mm][c]{3zw}{\centering{%
		{\fontsize{9pt}{9pt}\selectfont 新郎}\\[1mm]%
		{\fontsize{11pt}{11pt}\selectfont \kintou{2.5zw}{#4}}%
		}}\hskip15mm%
		\parbox[c][11mm][c]{3zw}{\centering{%
		{\fontsize{9pt}{9pt}\selectfont 新婦}\\[1mm]
		{\fontsize{11pt}{11pt}\selectfont \kintou{2.5zw}{#5}}%
		}}}\leavevmode\\[2mm]%
		\linethickness{0.3pt}
		\framebox(55,5){\fontsize{10pt}{10pt}\selectfont 高 砂}%この席名は不要?
		\end{minipage}%}%
		\hskip10.5mm%
		\begin{minipage}[c]{100mm}%
		\centering{%
		\fontsize{10pt}{10pt}\selectfont #6}%
		\end{minipage}%
	\end{minipage}}%}%
}
%%%%%%%%%%%%%%%%%%%%%%%%%%%%%%%%%%%%%%%%%%%%%%%%%%%%%%%%%%%%%%%%%%%%%%
%欄外備考
%%%%%%%%%%%%%%%%%%%%%%%%%%%%%%%%%%%%%%%%%%%%%%%%%%%%%%%%%%%%%%%%%%%%%%
\newcommand{\warning}[1]{%
%\put(146,0){%
\put(190,0){%
\begin{minipage}[t][8mm][c]{99mm}%
{\fontsize{7pt}{7pt}\selectfont #1}%
\end{minipage}}}
%%%%%%%%%%%%%%%%%%%%%%%%%%%%%%%%%%%%%%%%%%%%%%%%%%%%%%%%%%%%%%%%%%%%%%
\makeatother